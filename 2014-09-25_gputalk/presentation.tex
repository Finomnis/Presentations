\documentclass[t]{beamer}
 \setbeamercovered{transparent}

% Used Packages
\usepackage[T1]{fontenc}
\usepackage[utf8]{inputenc}
\usepackage{graphicx}
\usepackage{listings}

% The THEME
\usetheme{Madrid}
\setbeamertemplate{navigation symbols}{}

% Title Page
\title{Distributed GPGPU Computing}
\author{Martin Stumpf}
\institute{Ste||ar Group, Louisiana State University}
\date{\today}

% Title page before every section
\AtBeginSection[]
{
   \begin{frame}
       \frametitle{Outline}
       \tableofcontents[currentsection]
   \end{frame}
}

%%%% BEGIN OF THE ACTUAL DOCUMENT %%%%
\begin{document}

\frame{\titlepage}
\frame{\frametitle{Table of Contents}\tableofcontents}

\section{OpenCL}

\end{document}


     
- intro opencl
    - what is opencl?
        - open compute language
        - language for spmd problems
        - spmd?
#- intro gpu's
#    - what is a gpu?
#    - why are gpu's a good fit for opencl problems?
#    - lockstep - maybe
- distributed opencl - how?
    - MPI + OpenCL
        - Message System, data based
        - calculation, data exchange cycle
        - problem: implicit barrier between steps
        - problem: explicit distribution
- HPX        
        - what is HPX?
        - async RPC based
        - calculation, dependency tree
        - good fit for opencl, as opencl is in itself also dependency based
        - easy to program, use cluster as if it were a single machine
- HPXCL

- POCL?

